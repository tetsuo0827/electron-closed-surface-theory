
\documentclass[12pt]{article}
\usepackage{amsmath, amssymb}
\usepackage{geometry}
\geometry{a4paper, margin=1in}
\title{The Electron as a Resonant Closed Potential Surface: A Geometrical Reinterpretation of the Quantum Probability Cloud}
\author{Tetsuo Konno}
\date{}

\begin{document}
\maketitle

\begin{abstract}
This paper proposes a geometrical reinterpretation of the electron based on the hypothesis that the so-called probability cloud, as described in quantum mechanics, is not merely an abstract probabilistic distribution but rather the manifestation of a physically real closed potential surface. We argue that the wavefunction is constrained to this surface, and the electron itself should be identified with the resonant vibrational modes on this surface. This approach reconciles quantum uncertainty with geometrical determinism and offers a unified framework that bridges intuitive qualia-based thinking and abstract mathematical formalism.
\end{abstract}

\section{Introduction}
Conventional quantum mechanics describes the electron as a probability distribution derived from the square of a complex-valued wavefunction $\psi(x, t)$. This interpretation, while successful, raises ontological ambiguities about the true nature of the electron. Is it a particle, a wave, or merely information?

This paper suggests a novel interpretation: that the probability cloud reflects not just epistemic uncertainty, but the real structure of a closed surface shaped by the potential $V(x)$ in space. We posit that the electron is not located ``on a point'' but rather is the surface itself.

\section{Foundations: Probability and Potential Equi-surfaces}
We start by noting that the probability density $|\psi(x)|^2$ becomes physically meaningful only within the structure imposed by the external potential. In three-dimensional space, the level sets $\{x \mid V(x) = c\}$ form closed equipotential surfaces under many conditions (e.g., atomic potentials).

We redefine the electron's position not as a point in space, but as a mode of vibrational resonance constrained to such a closed surface.

\section{Wavefunction as Surface-Bound Resonance}
Let $S_c = \{x \in \mathbb{R}^3 \mid V(x) = c\}$ be a closed surface. We define a surface-bound wavefunction $\phi(s,t)$, where $s$ is a point on $S_c$, and propose the following:
\begin{itemize}
  \item The dynamics of $\phi$ follow a surface-restricted wave equation: $\partial^2 \phi / \partial t^2 = c_s^2 \Delta_{S} \phi$
  \item The quantized modes of $\phi$ correspond to electron energy levels
  \item The traditional $\psi(x)$ is a projection of $\phi(s)$ back into $\mathbb{R}^3$ via surface density mapping
\end{itemize}

\section{Implications and Interpretations}
This model implies:
\begin{itemize}
  \item The electron has an intrinsic geometrical identity
  \item Measurement-induced collapse corresponds to localized surface resonance convergence
  \item The classical ``electron cloud'' is a misinterpretation of a resonant field structure
\end{itemize}

\section{Discussion: From Qualia to Geometry}
This interpretation is not merely mathematical. It is rooted in perceptual intuition---we feel curvature, tension, and enclosure. These qualia correspond well to the proposed physical structure. Thus, the boundary between subjective intuition and objective physics becomes blurred in a fruitful way.

\section{Conclusion}
By reinterpreting the electron as a closed, resonant surface rather than a point-like particle or abstract wave, we offer a new path forward in the conceptual foundation of quantum mechanics---a path where geometry and probability cohere naturally.

\end{document}
